\documentclass{ctexart}

\setCJKmainfont[BoldFont=SimHei,ItalicFont=KaiTi]{SimSun}
\usepackage{tikz}
\usetikzlibrary{shapes,arrows}
\usepackage{amsmath} % nice math symbols
\usepackage{bm} % bold math
\usepackage{color} % change text color
%%%<
\usepackage{verbatim}
\usepackage[active,tightpage]{preview}
\PreviewEnvironment{tikzpicture}
\setlength\PreviewBorder{5pt}%

\begin{document}
	
	\tikzstyle{decision} = [diamond, draw,
	text badly centered,aspect=3, node distance=2cm, inner sep=0pt]%text width=4.5em,
	\tikzstyle{block} = [rectangle, draw,
	text centered, rounded corners, font=\small,minimum width = 6cm]%text width=5em,
	\tikzstyle{blockend} = [rectangle, draw,
	text centered, rounded corners, font=\small]%text width=5em,
	\tikzstyle{line} = [draw, -latex']
	%\tikzstyle{cloud} = [draw, ellipse,fill=red!20, node distance=3cm]
	%\tikzstyle{arrow1} = [ ->, >= stealth]
	%\tikzstyle{arrow2} = [ -, >= stealth]
		\begin{tikzpicture}[node distance = 1.2cm, auto]
		\node [blockend] (a1) {开始};
		\node [block,below of=a1,minimum width = 4cm] (a2) {申请变量空间,并初始化};
		\node [block,below of=a2,minimum width = 4cm] (a3){读入分网文件};
		\node [block,below of=a3,minimum width = 4cm] (nrstart){开始第$k$(k=0)次牛顿迭代};
		\node [block,below of=nrstart] (a4){计算迭代单元系数矩阵$S^e$};
		\node [block,below of=a4] (a5){计算雅可比矩阵$J^e$};
		\node [block,below of=a5] (a6){计算右侧矩阵$F^e$};
		\node [block,below of=a6] (a7){进行有限元装配};
		\node [block,below of=a7] (a8){求解$J_{k}A_{k+1}=F+(J_{k}-S_{k})A_{k}$};
		\node [block,below of=a8] (a9){计算单元中$\mu$和$\boldsymbol{B}$值};
		\node [decision,below of=a9,node distance=1.5cm] (error) {$
			\left| \frac{A_{k+1} - A_{k}}{A_{k}}    
			\right| < \varepsilon?
			$};
		
		\node [blockend,below of=error,node distance=1.8cm] (a10){求解结束};
		
		%通过放置两个看不见的node来绘制折线
		\node [right of=a4,node distance=4cm] (a11){};
		\node [right of=error,node distance=4cm] (a12){};
		
		\path [line] (a1) -- (a2);
		\path [line] (a2) -- (a3);
		\path [line] (a3) -- (nrstart);
		\path [line] (nrstart) -- (a4);
		\path [line] (a4) -- (a5);
		\path [line] (a5) -- (a6);
		\path [line] (a6) -- (a7);
		\path [line] (a7) -- (a8);
		\path [line] (a8) -- (a9);
		\path [line] (error) --node {YES} (a10);
		\path [line] (a9) -- (error);
		
		\path [line] (error) --node{NO} (a12.center) --node[above,rotate=90]{$k=k+1$} (a11.center)-- (a4);
		%\draw [arrow1] (a11.center) -- (a4);
		%\draw [arrow2] (error) --node{NO} (a12.center);
		%\draw [arrow2] (a12.center) -- (a11.center);
		\end{tikzpicture}
		
\end{document}