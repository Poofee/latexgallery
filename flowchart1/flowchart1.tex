\documentclass{ctexart}
%\usepackage{ctex}
\setCJKmainfont[BoldFont=SimHei,ItalicFont=KaiTi]{SimSun}
\usepackage{tikz}
\usetikzlibrary{shapes,arrows}
\usepackage{amsmath} % nice math symbols
\usepackage{bm} % bold math
\usepackage{color} % change text color
%%%<
\usepackage{verbatim}
\usepackage[active,tightpage]{preview}
\PreviewEnvironment{tikzpicture}
\setlength\PreviewBorder{5pt}%
%%%>

\begin{comment}
:Title: Simple flow chart
:Tags: Diagrams

With PGF/TikZ you can draw flow charts with relative ease. This flow chart from [1]_
outlines an algorithm for identifying the parameters of an autonomous underwater vehicle model.

Note that relative node
placement has been used to avoid placing nodes explicitly. This feature was
introduced in PGF/TikZ >= 1.09.

.. [1] Bossley, K.; Brown, M. & Harris, C. Neurofuzzy identification of an autonomous underwater vehicle `International Journal of Systems Science`, 1999, 30, 901-913


\end{comment}


\begin{document}
\pagestyle{empty}

%定义形状样式
\tikzstyle{startstop} = [rectangle, rounded corners, minimum width = 3cm, minimum height = 0.7cm, text centered, draw = black]
\tikzstyle{startstop2} = [rectangle, rounded corners, minimum width = 13cm, minimum height = 0.7cm, text centered, draw = black]
\tikzstyle{io} = [trapezium, trapezium left angle = 30, trapezium right angle = 150, minimum width = 3cm, text centered, draw = black, fill = white]
\tikzstyle{io2} = [trapezium, trapezium left angle = 30, trapezium right angle = 150, minimum width = 2.5cm, draw = black, fill = white]
\tikzstyle{io3} = [trapezium, trapezium left angle = 30, trapezium right angle = 150, minimum width = 2cm, draw = black, fill = white]
\tikzstyle{process} = [rectangle, minimum width = 3cm, minimum height = 1cm, text centered, draw = black]
\tikzstyle{decision} = [diamond, minimum width = 3cm, minimum height = 1cm, text centered, draw = black]
\tikzstyle{arrow} = [thick, -, >= stealth]
\tikzstyle{arrow2} = [thick, ->, >= stealth]

\begin{tikzpicture}[node distance = 1.5cm]
% 定义流程图具体形状
\coordinate[label = left:{\small 输入图像}](A) at(-1.5, 0);
\node(in1) [io] {};
\node(pro1) [startstop, below of = in1] {\small 线性滤波};

\node(in2 - 2)[io3, below of = pro1, yshift = -0.6cm]{};
\node(in3 - 2)[io3, left of = in2 - 2, xshift = -2.5cm]{};
\node(in4 - 2)[io3, right of = in2 - 2, xshift = 2.5cm]{};

\node(in2 - 1)[io2, below of = pro1, yshift = -0.3cm]{};
\node(in3 - 1)[io2, left of = in2 - 1, xshift = -2.5cm]{};
\node(in4 - 1)[io2, right of = in2 - 1, xshift = 2.5cm]{};

\node(in2) [io, below of = pro1] {\small 颜色};
\node(in3)[io, left of = in2, xshift = -2.5cm]{ \small 亮度 };
\node(in4)[io, right of = in2, xshift = 2.5cm]{ \small 方向 };

\node(in5)[startstop2, below of = in2 - 2]{ \small Center - Surround差异计算及归一化 };

\node(in6 - 2)[io3, below of = in5, yshift = -0.6cm]{};
\node(in7 - 2)[io3, left of = in6 - 2, xshift = -2.5cm]{};
\node(in8 - 2)[io3, right of = in6 - 2, xshift = 2.5cm]{};

\node(in6 - 1)[io2, below of = in5, yshift = -0.3cm]{};
\node(in7 - 1)[io2, left of = in6 - 1, xshift = -2.5cm]{};
\node(in8 - 1)[io2, right of = in6 - 1, xshift = 2.5cm]{};

\node(in6) [io, below of = in5] {};
\node(in7)[io, left of = in6, xshift = -2.5cm]{};
\node(in8)[io, right of = in6, xshift = 2.5cm]{};

\coordinate[label = left:{\small 特征图}](B) at(-1, -6.2);
\coordinate[label = left:{\small (12张)}](C) at(-1.5, -7.5);
\coordinate[label = left:{\small (6张)}](D) at(2.7, -7.5);
\coordinate[label = left:{\small (24张)}](E) at(6.7, -7.5);

\node(in9)[startstop2, below of = in6 - 2]{ \small 跨尺度合并及归一化 };

\node(in10) [io, below of = in9] {};
\node(in11)[io, left of = in10, xshift = -2.5cm]{};
\node(in12)[io, right of = in10, xshift = 2.5cm]{};

\coordinate[label = left:{\small 醒目图}](F) at(-1, -9.5);
\node(in13) [startstop, below of = in10] {\small 线性组合};
\node(in14) [io, below of = in13] {};
\coordinate[label = left:{\small 显著图}](G) at(-1, -13);

\node(in15) [startstop, below of = in14] {\small 赢者取全};
\coordinate[label = left:{\small 显著位置}]() at(1, -16.1);
\coordinate[label = left:{\small 反馈抑制}]() at(4.5, -14.7);

%连线
\draw[arrow](pro1) -- (in1);
\draw[arrow](pro1) -- (in2);
\draw[arrow](pro1) -- (in3);
\draw[arrow](pro1) -- (in4);
\draw[arrow](0, -4.75) -- (in2 - 2);
\draw[arrow](-4, -4.75) -- (in3 - 2);
\draw[arrow](4, -4.75) -- (in4 - 2);
\draw[arrow](0, -5.45) -- (in6);
\draw[arrow](-4, -5.45) -- (in7);
\draw[arrow](4, -5.45) -- (in8);
\draw[arrow](0, -8.35) -- (in6 - 2);
\draw[arrow](-4, -8.35) -- (in7 - 2);
\draw[arrow](4, -8.35) -- (in8 - 2);
\draw[arrow](0, -9.05) -- (in10);
\draw[arrow](-4, -9.05) -- (in11);
\draw[arrow](4, -9.05) -- (in12);
\draw[arrow](in13) -- (in10);
\draw[arrow](in13) -- (in11);
\draw[arrow](in13) -- (in12);
\draw[arrow](in13) -- (in14);
\draw[arrow](in14) -- (in15);
\draw[arrow](in15) -- (0, -15.8);
\draw[arrow](0, -15.4) -- (2.5, -15.4);
\draw[arrow](2.5, -14) -- (2.5, -15.4);
\draw[arrow2](2.5, -14) -- (0, -14);
\end{tikzpicture}


\end{document} 